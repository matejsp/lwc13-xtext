\subsection{Xtend}
 
Since we will use Xtend to write the code generator, this chapter describes some
basic concepts of the Xtend language. However, we will not cover all aspects of
Xtend in this section, for more information see also the official documentation\footnote{\url{http://www.eclipse.org/xtend/documentation.html}}.

Xtend is a statically-typed general purpose language similar to Java. Xtend uses
Xbase as core language which was already roughly explained in section \ref{sec:Xbase}.
Hence, the presented concepts of Xbase are also valid for Xtend. The main focus
of Xtend lies in providing a language that is more readable than Java in certain
situations. In the background, Xtend compiles to Java code. Thus, it plays
perfectly together with Java, e.g. methods declared in Java classes can be called in Xtend
and vice versa. Several concepts of Xtend are especially beneficial when writing code
generators. 

\begin{lstlisting}[language=Xtend]
def someMethod() {
  var myQuestion = 'where do we go?'
  myQuestion = myQuestion.toFirstLower
  val myAnswer = 42
  'The answer for question '+myQuestion+' is: '+myAnswer
}
\end{lstlisting}

In Xtend, a method is defined with the \texttt{def} keyword. The return type is
optional and will be automatically inferred. Only when a method is recursively
invoking itself, a return type needs to be specified explicitly. The keyword 
\texttt{var} defines a variable; constant values are defined with \texttt{val}.
In Xtend - since it is based on Xbase - everything is an expression, meaning that 
it has a return type. Consequently, the last expression of a method defines the
return value and also the return type of the method.

In line 3 a so-called extension method is used. Recall that the String class in 
Java does not provide a method \texttt{toFirstUpper}. Xtend allows for extending
closed types without changing them (maybe you already guess where Xtend got its
name from).
You can easily write your own extensions, e.g. for the Integer type:

\begin{lstlisting}[language=Xtend]
  def square(Integer input) {
  	input * input
  }
  
  def useExtension() {
  	16.square // returns 256
  }
\end{lstlisting}

What Xtend basically does is changing the syntax of how methods are called. Instead
of writing \texttt{toLastUpper(``Hello'')}, Xtend always offers the alternative to
use the first input parameter as receiver of the method call (\texttt{``Hello''.toLastUpper}).
This results in the syntax being more chained than nested which improves readability.
To use methods from another class as extension methods in your class, the field
defining an object of the other class needs to be marked with the \texttt{extension}
keyword. In our scenario we will use dependency injection like the following:

\begin{lstlisting}[language=Xtend]
class MyGenerator implements IGenerator{
  @Inject extension IJvmModelAssociations
  ...
}
\end{lstlisting}

This statement simply allows to use the methods declared by the interface
\newline\texttt{IJvmModelAssociations} in our generator class as extension methods
on our objects. Which concrete implementation of the interface is later actually called, is configured in the Guice module.

A further useful feature of Xtend is \emph{polymorphic dispatching}. Using the \texttt{dispatch}
keyword on multiple methods with identical signatures has the effect that the 
decision on which method should be called is based on the runtime type of the
target object. In contrast, Java binds methods at compile time based on the static
type of the target object. Since Xtend compiles to Java, polymorphic dispatching
is internally realized by a dispatcher method using a cascade of
\texttt{instanceof} constructs. 

\begin{lstlisting}[language=Xtend]
  def dispatch doSomething(SubTypeA input) {
  	// do something SubTypeA specific
  }
  
  def dispatch doSomething(SubTypeB input) {
  	// do something SubTypeB specific
  }
  
  def useDispatching() {
	val SuperType a = new SubTypeA()
	val SuperType b = new SubTypeB()
    a.doSomething + b.doSomething
  }
\end{lstlisting}

In this example the types \texttt{SubTypeA} and \texttt{SubTypeA} are sub types
of \texttt{SuperType}. The method \texttt{doSomething} is declared for each of
the two sub types as input parameter with accordingly different body (here
only indicated by comments). In line 12 these methods are called on the objects
\texttt{a} and \texttt{b} which have \texttt{SuperType} as static type and one
of the sub types as compile time type. Xtend invokes the correct method here,
i.e. for \texttt{a} the method in lines 1-3 and for \texttt{b} the one in lines
5-7 is called.

Xtend offers the possibility to define \emph{Rich Strings} (also called \emph{templates})
which is especially useful when writing code generators. Rich Strings allow for
writing complex Strings with line breaks and indentations without the need for
concatenating special characters like \texttt{'\textbackslash n'} or \texttt{'\textbackslash t'}. A rich String
construct starts and ends with triple single quotes
(\textquotesingle\textquotesingle\textquotesingle) Within such a String code
pieces which themself return a String can be inserted (surrounded by guillemots
\texttt{\guillemotleft\guillemotright}).
There are also logical structures like \texttt{for} loops or \texttt{if-else}
statements supported:

\begin{lstlisting}[language=Xtend]
  def htmlContent(List<String> contents) '''
  	<html>
  	  <body>
  	    «FOR content: contents BEFORE '<p>' SEPARATOR '<br/>' AFTER '</p>'»
  	      «IF content.length > 10»
  	        Large Content: «content.toFirstUpper»
  	      «ELSE»
  	        Small Content: «content.toFirstUpper»
  	      «ENDIF»
  	    «ENDFOR»
  	  </body>
  	</html>'''
\end{lstlisting}

Note the special keywords in the \texttt{FOR} loop declaration: \texttt{BEFORE} 
and \texttt{AFTER} will be called once before and after the iteration, but only
if the loop will be iterated at least once. With the \texttt{SEPARATOR} keyword
a string which will be inserted between two iteratins can be specified. Calling 
the example method with ['hello', 'Some more text'] results in:

\begin{lstlisting}[language=Xtend]
<html>
  <body>
    <p>
    Small Content: Hello<br/>
    Large Content: Some more text
    </p>
  </body>
</html>
\end{lstlisting}

Last but not least, Xtend offers a more sophisticated \texttt{switch-case} statement
than Java does. The \texttt{break} statement as known from Java is implicit in Xtend.
Furthermore, switching based on Strings and even based on the type of the switch
argument is possible, like in the following example:

\begin{lstlisting}[language=Xtend]
  def getTypeName(Number input) {
  	switch (input) {
  		Integer: "It is an Integer!"
  		Float:   "It is a Float!"
  		default: "It is some other number type."
  	}
  }
\end{lstlisting}

Now as you know the basic concepts of Xtend, let's finally start writing the code
generator.
