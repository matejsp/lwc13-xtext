\subsection{Xbase}

The language developed in section \ref{sec:DefiningGrammar} does not yet meet all 
demands on the LWC2013 task. Two core features are missing: First, a question's 
answer can be computed, i.e. its answer can be derived from an expression referring 
to previous questions' answers. Second, questions can be optional depending on the 
previous answers. For this, also the possibility to define expressions is needed. This
is where Xbase comes into play.

Xbase is an expression language which can be reused in your own Xtext DSL. Its language
concepts are similar to Java, but with some syntactical derivations improving readability.
The Xbase grammar is defined in Xtext, thus its elements can be used in any other Xtext grammar
by importing or directly extending Xbase via Xtext's possibility for grammar inheritance.
In addition to the grammar, Xbase ships with further infrastructural parts like a compiler, 
interpreter, linker or static analyzer which all can be adapted to your own needs. In
the background, Xbase produces plain Java code which is run on the JVM. Like
other DSLs defined with Xtext, Xbase provides also editor features like syntax highlighting, 
content assistance and navigation via hyperlinks. In the following we will first introduce
some language concepts of Xbase, and afterwards we will describe how to integrate Xbase
into the Questionnaire DSL.

In Xbase everything is an expression which always has a return type which might be \texttt{null} 
for some expressions. Variables are defined with the \texttt{var} keyword, whereas for 
constant values the \texttt{val} keyword is used. Types are derived automatically, so they 
don't need to be defined explicitly:

\begin{lstlisting}[language=Xbase]
var myVariable = 'some modifiable value'
val Integer myConstant = 42  
\end{lstlisting}

Xbase ships with a library extending existing Java types like String or Integer with further 
functionality. So besides the already known String operations from Java like 
\texttt{toUpperCase} or \texttt{toLowerCase}, in Xbase expressions you can also use \texttt{toFirstUpper}
and \texttt{toFirstLower} changing only the first letter's case which might come in handy 
in some situations. Large numbers can be written more readable by using underscores to
separate digits:

\begin{lstlisting}[language=Xbase]
"a day has ".toFirstUpper() + 86_400_000 + " milliseconds."
// results in: A day has 86400000 milliseconds.
\end{lstlisting}

As in Java, Xbase provides \texttt{if-else}-expressions for defining conditions. Since each expression
has a return type, it is valid to use  \texttt{if-else}-blocks similar to the ternary operator in Java:

\begin{lstlisting}[language=Xbase]
var x = if (condition) 42 else 43
\end{lstlisting}

There are further concepts in Xbase which we will not cover here in more detail, since they have
not much relevance for the Questionnaire language. So e.g. it is possible to use loops for
iterating over a collection of element; there is a \texttt{switch-case}-expression
with type guards allowing for defining behavior depending on the type of a parameter; and last but
not least, Xbase allows the definition of closures. For more details, please look up the reference
documentation\footnote{\url{http://www.eclipse.org/Xtext/documentation.html\#xbaseLanguageRef\_Introduction}}
or the Xbase tutorials directly in Eclipse (\emph{File / New / Other.. / Xbase Tutorial}). 

With these capabilities integrated in the Questionnaire language it is feasible to define complex
domain logic e.g. for the result of a questionnaire directly in its definition. For example, when
designing a questionnaire for a test, let's say to define a person's stress level, you can write
some Xbase code as expression for the last ``result'' question:

\begin{lstlisting}[language=Xbase]
stressLevelResult: "Your Stress-Level: " String (
	{
		var Integer stressPoints = if (hasTimePressureAtWork) 30 else 0
		stressPoints = stressPoints + daysSleepingBadPerWeek * 3
		stressPoints = stressPoints + glassesOfAlcoholPerDay * 12
		stressPoints = stressPoints - daysWithSportPerWeek * 2
		if (stressPoints>80) "High" else if (stressPoints>40) "Medium" else "Low"
	}
) 
\end{lstlisting}
