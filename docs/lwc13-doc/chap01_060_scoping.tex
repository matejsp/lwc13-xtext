\subsection{Scoping} \label{sec:scoping}

Scoping is, roughly said, the computation of referrable names in a given
context. It is a quite complex topic, and we won't cover it here into deep.
The topic itself is heavily documented by the Xtext user
manual\footnote{\href{http://www.eclipse.org/Xtext/documentation.html\#scoping}{Xtext
manual:Scoping}}, several articles\footnote{e.g.
\url{http://blogs.itemis.de/stundzig/archives/776}} and implementation examples
\footnote{e.g. \url{https://github.com/LorenzoBettini/xtext-scoping}}.

The Xtext framework already provides default implementations to solve scoping
and linking. In the case of Xbase based languages the
\texttt{XbaseBatchScopeProvider} is configured by default as implementation of
the \texttt{IScopeProvider} interface.

For our use case, the default behavior is already sufficient. Through the
JVM model inference the implicit \texttt{``this''} variable is already bound to the class
representing the form, and thus the fields representing the question elements
are known as callable features.

