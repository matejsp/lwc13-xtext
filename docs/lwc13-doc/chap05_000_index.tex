\section{Closing Words}

Thank you for reading this document. We have been writing it with the intention
that it should provide easy access for first-time users of Xtext to this
powerful Language Workbench. This explains also the size of the document. We
could have written it shorter, if we assumed more background knowledge of
the potential readers, or if we left many things with just short
comments.

Xtext has grown over years, gaining more and more
experience from real-life projects that want to leverage the power of DSLs in
integrated environments. We could only touch the surface of Xtext and Xtend
here, and tried to choose the most simple solution. The QL assignment did fit
quite well into what Xtext can provide mostly out-of-the-box, but its real power
is unvealed when DSLs have non-standard requirements. Almost every peace in
Xtext is highly customizable and it is one of the most flexible frameworks we
know of. This flexibility comes to the price of complexity. Xtext is well
documented by the reference manual, several blogs show advanced concepts in
detail, and hundreds of projects solve different real-world requirements. Many
open projects exist which are worth studying. Learning Xtext does not mean
following a single tutorial, it needs training. The sources and the debugger are
valuable friends when solving issues. The community around Xtext is very helpful
and likely the largest of all language workbenches.

We hope that this document was helpful for you to understand some concepts of
Xtext and gave you the right level of abstraction to learn how to use this tool.
If we helped you to decide to use Xtext or Xtend in your project, then it was
worth the effort and we would be pleased to hear from you! 

\par
\par
\par

\emph{\large{Karsten Thoms, Johannes Dicks, Thomas Kutz}}

\emph{April 2013}

