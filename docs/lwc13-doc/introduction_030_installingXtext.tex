\subsection{Installing Eclipse & Xtext}

Xtext is a SDK for the \href{http://www.eclipse.org/}{Eclipse} IDE. To
install it you have two options:

\begin{compactitem}
    \item You can download Xtext separately and install it in your Eclipse
    instance.
    \item You can download a specially-crafted complete Eclipse distribution
    which has Xtext prepackaged already.
\end{compactitem}

We will take the latter approach here and describe the individual steps:

\begin{compactenum}
    \item Go to the \href{http://download.itemis.com/distros/}{Xtext download
    page}. Here you can get Eclipse 4.2.x (Helios) including Xtext 2.3.1
    along with some tools Xtext depends on. The latter are subsumed here under
    ``Xtext'' for simplicity.
    If you want you can download also a distribution
    which is already bundled with Eclipse 4.3.0 Kepler, but be aware that this
    is not finalized until end of June 2013.
    \item The Eclipse/Xtext distribution is available for multiple platforms.

    \begin{compactenum}
      \item
      \href{http://download.itemis.com/distros/eclipse-SDK-4.2-Xtext-2.3.1-linux-gtk-x86_64.tar.gz}{Linux GTK x86 64 bit}
      \item
      \href{http://download.itemis.com/distros/eclipse-SDK-4.2-Xtext-2.3.1-linux-gtk.tar.gz}{Linux GTK x86 32 bit}
        \item
        \href{http://download.itemis.com/distros/eclipse-SDK-4.2-Xtext-2.3.1-macosx-cocoa-x86_64.tar.gz}{Mac OSX x86 64 bit}
        \item
        \href{http://download.itemis.com/distros/eclipse-SDK-4.2-Xtext-2.3.1-win32-x86_64.zip}{Windows 64 bit}
        \item
        \href{http://download.itemis.com/distros/eclipse-SDK-4.2-Xtext-2.3.1-win32.zip}{Windows 32 bit}
    \end{compactenum}

    \item Unpack the downloaded archive file in a directory of your choice.

        Example (Linux):

\begin{lstlisting}
  cd /opt/local
  gzip -dc /download/eclipse-SDK-4.2-Xtext-2.3.1-linux-gtk-x86_64.tar.gz | tar
  xvfp -
\end{lstlisting}

        The archive will be extracted to a new directory named \texttt{eclipse}. Before
        unpacking the archive, please ensure that there is no subdirectory named
        \texttt{eclipse} yet! Different operating systems may require different unpacking
        methods.\footnote{On Windows do not unpack it into a deep directory,
        since this might cause troubles with long path names.}

    \item Start Eclipse by running the \texttt{eclipse} executable in the newly-created
    \texttt{eclipse} directory.
\end{compactenum}
