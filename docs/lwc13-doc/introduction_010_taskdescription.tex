\subsection{Task Description}
The {\href{http://www.languageworkbenches.net/images/5/53/Ql.pdf}{LWC13 task}}
is to implement a DSL for questionnaires (Questionnaire Language, QL), which
basically allows the definition of forms with questions.


\subsection{Technology Stack}
This tutorial expects that you are somehow familiar with Java and Eclipse and
have heard about \url{EMF} and how it works in general before. We start almost at the
beginning, but not quite :-) 

We will use Xtext 2.3.1, which is at the moment of writing the latest official
release.
Xtext 2.4 is in preparation and will be released with Eclipse Kepler in June
2013\footnote{\url{http://wiki.eclipse.org/Kepler/Simultaneous_Release_Plan}}.
The solution approach described here would work also with any version
of Xtext >= 2.0, but the API might differ slightly, so there is no guarantee
that each codeline printed here would work exactly with all versions. For better
reproduction it is highly recommended to use the versions mentioned above.

For Code Generation we will use the language Xtend, which itself is based on
Xtext. Xtend makes use of a common expression language shipped with Xtext called
Xbase. The languages developed here will also be based on Xbase, but more on
this later.

The reference implementation of the Xtend generator will generate, 
JavaServer Faces 2.1( JSF).\footnote{\url{http://www.javaserverfaces.org/}} 
JSF is part of the Java Enterprise Edition (Java EE). It is useful to have a 
basic understanding of how web applications work even if JSF provides a nice level 
of abstraction. The JSF reference implementation from 
Oracle Mojarra 2.1.6\footnote{\url{http://javaserverfaces.java.net/}} is able to run 
within the well known Servlet container Apache Tomcat( v7.0).\footnote{\url{http://tomcat.apache.org/}} 

To get a nicely integrated developement environment we will install some components of the
Web Tools Platform( WTP)\footnote{\url{http://www.eclipse.org/webtools/}} into an existing Eclipse installation.