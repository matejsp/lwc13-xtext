The tasks to solve for the Language Workbench Challenge is relatively trivial.
But it has been designed to illustrate as many properties of language
workbenches as possible. As such, it can serve as a good introduction to the
respective tools.

This tutorial expects that you are somehow familiar with Java and Eclipse and
have heard about \url{EMF} and how it works in general before. We start almost at the
beginning, but not quite :-)

For this tutorial we will use Xtext 2.4, which is at the moment of writing not
finalized, but in its latest stage. Xtext 2.4 will be released with Eclipse
Kepler in June
2013\footnote{\url{http://wiki.eclipse.org/Kepler/Simultaneous_Release_Plan}}.


\subsection{Task Description}
The LWC13 task{\url{http://www.languageworkbenches.net/images/5/53/Ql.pdf}} is
to implement a DSL for questionnaires (Questionnaire Language, QL), which
basically allows the definition of forms with questions.  

%\subsubsection{Phase 0 - Basics}
%
%\begin{compactitem}
%    \item 0.1 Simple (structural) DSL without any fancy expression language or
%    such. Build a simple data definition language to define entities with
%    properties. Properties have a name and a type. It should be possible to use
%    primitive types for properties, as well as other Entities.
%    \item 0.2 Code generation to GPL such as Java, C\#, C++ or XML. Generate
%    Java Beans (or some equivalent data structure in C\#, Scala, etc.) with
%    setters, getters and fields for the properties.
%    \item 0.3 Simple constraint checks such as name-uniqueness For example,
%    check the name uniqueness of the properties in the entities.
%    \item 0.4 Show how to break down a (large) model into several parts, while
%    still cross-referencing between the parts.
%\end{compactitem}
%
%\subsubsection{Phase 1 - Advanced}
%
%This phase demonstrates advanced features not necessarily available to the same
%extent in every LWB.
%
%\begin{compactitem}
%    \item 1.1 Show the integration of several languages. Define a second
%    language to define instances of the entities, including assignment of values
%    to the properties. This should ideally really be a second language that
%    integrates with the first one, not just ``more syntax'' in the same grammar.
%    We want to showcase language modularity and composition here.
%    \item 1.2 Demonstrate how to implement runtime type systems. The
%    initialization values in the instance-DSL must be of the same type as the
%    types of the properties.
%    \item 1.3 Show how to do a model-to-model transformation. Define an ER-meta
%    model (Database, Table, Column) and transform the entity model into an
%    instance of this ER meta model.
%    \item 1.4 Some kind of visibility/namespaces/scoping for references.
%    Integrate namespaces/packages into the entity DSL and make sure in the
%    instance DSL you can only assign values to the properties of the respective
%    entity.
%    \item 1.5 Integrating manually written code (again in Java, C\# or C++).
%    Integrate derived attributes to entities. Note that if you want, you can
%    also define or reuse an expression language that allows defining the
%    algorithm for calculating the age directly in the model. Ideally, you will
%    show both (manually written 3GL code as well as an expression language).
%    \item 1.6 Multiple generators. Generate some kind of XML structure from the
%    entity model.
%\end{compactitem}
%
%\subsubsection{Phase 2 - Non-Functional}
%
%Phase 2 is intended to show a couple of non-functional properties of the LWB.
%The task outlined below does not elaborate on how to do this.
%
%\begin{compactitem}
%    \item 2.1 How to evolve the DSL without breaking existing models
%    \item 2.2 How to work with the models efficiently in the team
%    \item 2.3 Demonstrate Scalability of the tools
%\end{compactitem}
%
%\subsubsection{Phase 3 - Freestyle}
%
%Every LWB has its own special ``cool features''. In phase three we want the
%participants to show off these features. Please make sure, though, that the
%features are built on top of the task described below, if possible.
