To get simple access to all generated forms in the application
 we want to generate an index page where a link is included for each form which is defined
in our model.
The generated index page should be saved in a file called index.xhtml within a
subfolder \texttt{'generated/forms/'} of the \texttt{WEB\_CONTENT}
OutputConfiguration created in section \ref{sec:outputConfigurationProvider}.

\begin{lstlisting}[language=Java] 
class JSFGenerator implements IGenerator{
  @Inject extension JSFOutputConfigurationProvider
  @Inject extension QlDslExtensions
  
  override doGenerate(Resource input, IFileSystemAccess fsa) {
        if (input.URI.fileExtension!="ql")
            return
		// model root
        val questionnaire = input.contents.head as Questionnaire
		// generate index page with links to generated forms
        val contentIndex  = generate_FormIndex(questionnaire.forms)
        val fileNameIndex = "generated/forms/index.xhtml"
        fsa.generateFile(fileNameIndex,WEB_CONTENT, contentIndex)
        ...
  }
  ...
 }
      
\end{lstlisting}

For the new artifact add a new extension called \texttt{def
generate\_FormIndex}.
It receives a list of \texttt{Form} elements. In a short loop it generates an
outputlink for each form of the given model. Because \texttt{doGenerate} is
called for each QL resource the generator currently has the limitation that all
QL model elements are defined within the same QL resource to ensure generation of a
correct form index page.

\begin{lstlisting}[language=HTML] 
  def generate_FormIndex (List<Form> forms)
  '''...
      <ui:composition template="/index.xhtml">
        <ui:define name="content">
        
        «FOR elem: forms SEPARATOR "<br/>"»
          <h:outputLink value="«elem.name».jsf">«elem.name»</h:outputLink>
        «ENDFOR»
        
        </ui:define>
      </ui:composition>
      ... 
  '''
\end{lstlisting}

By using the attribute \texttt{template} of a \texttt{composition} tag as
already described in \ref{subsec:referenceMainLayout}, the structure and styles of
index.xhtml will be derived in our form index page. Our generated form index
needs a \texttt{index.xhtml} in the applications root folder which itself or one of its
parent templates defines a \texttt{facelet:insert} section with name 'content'
like described in section \ref{subsec:referenceMainLayout}.

To get the possibility to change the template for all generated files in
a single file easily later we will use the generated form index as template for
generated form pages in a later step. The generated form index looks similar to
the one described in section \ref{subsec:referenceForms}.
