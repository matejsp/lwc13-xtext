\subsubsection{Main Layout}
\label{subsec:referenceMainLayout} 

The logical entry to the web application is the \texttt{welcome-file}
(\texttt{WebContent/index.xhtml}) declared in
\texttt{WebContent/WEB-INF/web.xml}. The \texttt{index.xhtml} composes the
main layout with page contents and will later be helpful to integrate the
generated artifacts with the web applications layout by use of JSF xhtml templating.
\footnote{\url{http://docs.oracle.com/javaee/6/javaserverfaces/2.1/docs/vdldocs/facelets/}}

\begin{lstlisting}[language=HTML]
<?xml version='1.0' encoding='UTF-8' ?>
<!DOCTYPE html PUBLIC "-//W3C//DTD XHTML 1.0 Transitional//EN" 
    "http://www.w3.org/TR/xhtml1/DTD/xhtml1-transitional.dtd">
<html xmlns="http://www.w3.org/1999/xhtml"
  xmlns:h="http://java.sun.com/jsf/html"
  xmlns:ui="http://java.sun.com/jsf/facelets">
<body>

  <ui:composition
    template="/resources/default/templates/defaultLayout.xhtml">
    <ui:define name="content">
      Hello World!
    </ui:define>
  </ui:composition>

</body>

</html>
\end{lstlisting}
 
Subpages of the application should use the \texttt{index.xhtml} placed in
\texttt{WebContent/} itself as their template and overwrite the content section
with custom output by the same pattern.

\begin{lstlisting}[language=HTML] 
	<ui:composition template="/index.xhtml">
  		<ui:define name="content">
  		...my xhtml content
  		</ui:define>
 	</ui:composition>
\end{lstlisting}

Everything between the opening and closing \texttt{facelet:define} tag within
the subpage will be passed into a \texttt{facelet:insert} section with
\texttt{name = content} defined in the template
(here:\texttt{defaultLayout.xhtml}) or one of its parent templates when the html
output is rendered by the JSF framework.

\paragraph{Default template}
$\;$ \\
We keep the main layout definition and the page contents seperated from each
other. The \texttt{WebContent/index.xhtml} creates the main composition of the
applications structural layout and content of pages as described in section
\ref{subsec:referenceMainLayout}. Layout template defintions should be placed in
a folder \newline \texttt{WebContent/resources/*templateName*} in our web
application.

To change the main layout it is just necesarry to change the \texttt{template}
reference of the \newline\texttt{facelets:composite} in
\texttt{WebContent/index.xhtml}.\newline

\begin{lstlisting}[language=HTML]
 ...
  <ui:composition
    template="/resources/default/templates/defaultLayout.xhtml">
 ...
\end{lstlisting}

The reference application is shipped with a default template placed in \newline
\texttt{WebContent/resources/default/}.\newline Its a very simple one which
provides a skeleton \newline \texttt{/templates/defaultLayout.xhtml} \newline
with mainly 3 sections (header, content, footer) where clients can add custom content.

The current web application expects a defined \texttt{facelets:insert}
section with name 'content' within the template or one of its parents for proper
composition. In our refernence implementation it is declared in \newline
\texttt{/resources/default/templates/defaultLayout.xhtml}.

\begin{lstlisting}[language=HTML]
<!DOCTYPE html PUBLIC "-//W3C//DTD XHTML 1.0 Transitional//EN" 
          "http://www.w3.org/TR/xhtml1/DTD/xhtml1-transitional.dtd">
<html xmlns="http://www.w3.org/1999/xhtml"
	xmlns:h="http://java.sun.com/jsf/html"
	xmlns:ui="http://java.sun.com/jsf/facelets">
<h:head>
	<title><ui:insert name="title">LWC 2013 Xtext</ui:insert></title>
</h:head>
<body>
	<div id="header">
		<ui:insert name="header">
			<ui:include src="/resources/default/templates/header.xhtml" />
		</ui:insert>
	</div>
	<div id="content">
		<ui:insert name="content">
    	Content area. Compose by use of tag facelet:define & name="content".
  </ui:insert>
	</div>
	<div id="footer">
		<ui:insert name="footer">
			<ui:include src="/resources/default/templates/footer.xhtml" />
		</ui:insert>
	</div>
</body>
</html>
\end{lstlisting}

We added two \texttt{facelets:insert} sections to give the possibility to
replace the header and footer. To simplify the concrete JSF compositions later we let JSF include
default content for both by use of \texttt{facelets:include} for a fixed source
\texttt{src}.

\paragraph{Default CSS}
$\;$ \\
\ldots TODO nach erhalt der sourcen
