\subsection{JVM Model Inference}

For languages using Xbase it is necessary to tell Xtext, how to map concepts of a language to a Java model. In our example,
a Form could be mapped to the Type concept, while Questions are the fields of a class. By doing this, elements of the language
can be made available in expressions. Further, it allows that model elements are linkable where Java types are expected, without
necessarily generate a Java class.

The derivation of the Java model for language concepts is the responsibility of the JVM Model Inferrer, which is a class that implements
the \href{http://download.eclipse.org/modeling/tmf/xtext/javadoc/2.3/org/eclipse/xtext/xbase/jvmmodel/IJvmModelInferrer.html}{IJvmModelInferrer} interface.
A skeleton has already been generated into package org.eclipse.xtext.example.ql.jvmmodel. The file QlDslJvmModelInferrer.xtend is a class
written with Xtend.


The mapping that has to be implemented for the Questionnaire DSL should be as follows:
\begin{enumerate}
  \item Each Form instance is mapped to a JvmDeclaredType (which is the common concept for Java classes and interfaces).
  The type's name is simply the form name, and the target package is forms.
  \item Each Question of a Form is mapped to a JvmField, which is added as member of the declared type
  \item For each Question accessor methods for the field are generated. The field gets only a Setter, when the value of the Question is
  not computed by an expression. If the field is computed, the content of the getter has to compute the result.
  \item For
\end{enumerate}

- Describe the role of JVM Types
- Describe the role of the JVM Model Inferrer
- Implement JVM Model Inferrer
