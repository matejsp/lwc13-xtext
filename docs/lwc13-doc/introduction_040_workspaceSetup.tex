\subsection{Workspace Setup}

Before we begin, start Eclipse and set up a fresh workspace.

Some settings should be done. Open the workspace settings:

\begin{compactitem}
    \item Windows: Window / Preferences
    \item Mac: Eclipse / Preferences
\end{compactitem}

\paragraph{Workspace Encoding}
$\;$ \\
File encoding is important to some type of files. It is better that the
workspace is set to a common encoding to avoid any platform specific encoding.
By default the workspace is using platform encoding, which is Cp1252 on Windows
and MacRoman on Mac. We will use ISO-8859-1 as a common encoding here.

\begin{compactitem}
    \item Open Eclipse Preferences and go to \emph{General / Workspace}
    \item Change setting \emph{Text file encoding} to \emph{Other /
    ISO-8859-1}
\end{compactitem}

\paragraph{Launch Operation}
$\;$ \\

\begin{compactitem}
    \item Open Run/Debug / Launching
    \item Change ``Launch Operation'' to ``Always
    launch the previously launched application''
\end{compactitem}
This will allow you re-running the previous launched application by just
pressing the Run or Debug button in the Eclipse toolbar, or using keyboard
shortcuts. The default settings does not always do what you want.

